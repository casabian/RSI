% !TEX TS-program = pdflatex
% !TEX encoding = UTF-8 Unicode

% This is a simple template for a LaTeX document using the "article" class.
% See "book", "report", "letter" for other types of document.

\documentclass[11pt, a4paper]{article} % use larger type; default would be 10pt
\usepackage{a4wide}
\usepackage[utf8]{inputenc} % set input encoding (not needed with XeLaTeX)

%%% Examples of Article customizations
% These packages are optional, depending whether you want the features they provide.
% See the LaTeX Companion or other references for full information.

%%% PAGE DIMENSIONS
\usepackage{geometry} % to change the page dimensions
\geometry{a4paper} % or letterpaper (US) or a5paper or....
% \geometry{margin=2in} % for example, change the margins to 2 inches all round
% \geometry{landscape} % set up the page for landscape
%   read geometry.pdf for detailed page layout information

\usepackage{graphicx} % support the \includegraphics command and options

%\usepackage[parfill]{parskip} % Activate to begin paragraphs with an empty line rather than an indent

%%% PACKAGES
\usepackage{booktabs} % for much better looking tables
\usepackage{array} % for better arrays (eg matrices) in maths
\usepackage{paralist} % very flexible & customisable lists (eg. enumerate/itemize, etc.)
\usepackage{verbatim} % adds environment for commenting out blocks of text & for better verbatim
\usepackage{subfig} % make it possible to include more than one captioned figure/table in a single float
% These packages are all incorporated in the memoir class to one degree or another...

%%% HEADERS & FOOTERS
\usepackage{fancyhdr} % This should be set AFTER setting up the page geometry
\pagestyle{fancy} % options: empty , plain , fancy
\renewcommand{\headrulewidth}{0pt} % customise the layout...
\lhead{}\chead{}\rhead{}
\lfoot{}\cfoot{\thepage}\rfoot{}

%%% SECTION TITLE APPEARANCE
\usepackage{sectsty}
\allsectionsfont{\sffamily\mdseries\upshape} % (See the fntguide.pdf for font help)
% (This matches ConTeXt defaults)

%%% ToC (table of contents) APPEARANCE
\usepackage[nottoc,notlof,notlot]{tocbibind} % Put the bibliography in the ToC
\usepackage[titles,subfigure]{tocloft} % Alter the style of the Table of Contents
\renewcommand{\cftsecfont}{\rmfamily\mdseries\upshape}
\renewcommand{\cftsecpagefont}{\rmfamily\mdseries\upshape} % No bold!

\usepackage{natbib}
%%% END Article customizations

%%% The "real" document content comes below...

\title{Task 2 - References}
\author{Sebastian Geschke}
%\date{} % Activate to display a given date or no date (if empty),
         % otherwise the current date is printed 

\begin{document}
\maketitle
\section{Topic description}
The research paper should research the nature phenomenon of fluid dynamics, for application in real-time simulation and games. An implementation could be realized using a real-time fluid solver based on fast Fourier transform (FFT) or discrete sine/cosine transform (DCT/DST). 
\section{Reference summaries}

\subsection{\cite{Cohen:2010}: Interactive fluid-particle simulation using translating eulerian grids}
In the paper an interactive system featuring fluid-driven animation is presented. The system descriped includes a GPU-accelerated Eulerian fluid solver. In addition a hardware-accelerated volume rendering system to visualize the fluid dynamics is presented.

\subsection{\cite{daS.Junior:2011}: Two-way real time fluid simulation using a heterogeneous multicore CPU and GPU architecture}
This paper introduces a heterogeneous multicore CPU and GPU scalable architecture for fluid simulation with two-way interaction with solid objects. 
They also show the impacts of the architecture over GPU and CPU bounded simulations and provide results that can reproduce complex fluid behavior in real-time applications.

\subsection{\cite{Ebert:1994}: Volume rendering methods for computational fluid dynamics visualization}
The papaer describes three alternativ volume rendering approaches for visualizing fluid dynamics. The techniques include, volumetric gas rendering, a ray casting approach based on sampler illumination and a simple illumination model for rendering.

\subsection{\cite{Kellomaki:2012}: Water simulation methods for games: a comparison}
In this paper fluid simulation methods are compared, particularly with regard to actual game usage. They suggest the extremely simple but fast pipe model, where the simplicity of the underlying simulation can be masked using simple shader effects.

\subsection{\cite{Long:2009}: Real-time fluid simulation using discrete sine/cosine transforms}
This paper presents an fluid simulation system that is real-time capable. They provide an improvement to the Furier based solutions, which have inconsistencies near the boundaries, by solving the mass conservation step by using sine and cosine transforms.

\subsection{\cite{Nguyen:2007}: GPU Gems 3 - Chapter 30. Real-time simulation and rendering of 3D fluids}
The paper presented in GPU gems 3 provides a GPGPU implementation of a fluid solver that can be seamlessly integrated into any real-time application.

\subsection{\cite{Stam:2002}: A simple fluid solver based on FFT}
In this paper Stam presents a simple implementation of a fluid solver. He points out, that the mathematical reason to use Fourier transforms is that, mass-conserving and gradient fields are so simple in the Fourier domain follows from the fact that differentiation in the spatial domain corresponds to a multiplication by the wavenumber in the Fourier domain.

\subsection{\cite{Stam:2003}: Real-time fluid dynamics for games}
This paper presents a simple and rapid implementation of a fluid dynamics solver for game engines. The provided algorithms 
are based on the physical equations of fluid flow, namely the Navier-Stokes equations. 
The key points of the papers are, a fluid solver with roughly 150 lines of code, a good viusal quality and with reasonable grid sizes in two and three dimensions.

\subsection{\cite{Tu:2007}: Computational fluid dynamics: A practival approach}
This book provides a first course in computational fluid dynamics (CFD), where core mathematics are developed in order to  show the principles of the conservation laws, mathematical transport equations and basic concepts of fluid mechanics.

\subsection{\cite{Muller:2003}: Particle-based fluid simulation for interactive applications}
This paper introduces an interactive method based on Smoothed Particle Hydrodynamics (SPH) to simulate fluids with free surfaces.The paper shows that in contrast to Eulerian grid-based approaches, the particle-based approach makes mass conservation equations and convection terms disepnsable which significantly reduces the computational complexity of the simulation.
\bibliographystyle{chicago}
\bibliography{bibliography}

\end{document}
