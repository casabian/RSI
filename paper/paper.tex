% !TEX TS-program = pdflatex
% !TEX encoding = UTF-8 Unicode

% This is a simple template for a LaTeX document using the "article" class.
% See "book", "report", "letter" for other types of document.

\documentclass[11pt, a4paper]{article} % use larger type; default would be 10pt
\usepackage{a4wide}
\usepackage[utf8]{inputenc} % set input encoding (not needed with XeLaTeX)

%%% Examples of Article customizations
% These packages are optional, depending whether you want the features they provide.
% See the LaTeX Companion or other references for full information.

%%% PAGE DIMENSIONS
\usepackage{geometry} % to change the page dimensions
\geometry{a4paper} % or letterpaper (US) or a5paper or....
% \geometry{margin=2in} % for example, change the margins to 2 inches all round
% \geometry{landscape} % set up the page for landscape
%   read geometry.pdf for detailed page layout information

\usepackage{graphicx} % support the \includegraphics command and options

\usepackage[parfill]{parskip} % Activate to begin paragraphs with an empty line rather than an indent

%%% PACKAGES
\usepackage{booktabs} % for much better looking tables
\usepackage{array} % for better arrays (eg matrices) in maths
\usepackage{paralist} % very flexible & customisable lists (eg. enumerate/itemize, etc.)
\usepackage{verbatim} % adds environment for commenting out blocks of text & for better verbatim
\usepackage{subfig} % make it possible to include more than one captioned figure/table in a single float
% These packages are all incorporated in the memoir class to one degree or another...

%%% HEADERS & FOOTERS
\usepackage{fancyhdr} % This should be set AFTER setting up the page geometry
\pagestyle{fancy} % options: empty , plain , fancy
\renewcommand{\headrulewidth}{0pt} % customise the layout...
\lhead{}\chead{}\rhead{}
\lfoot{}\cfoot{\thepage}\rfoot{}

%%% SECTION TITLE APPEARANCE
\usepackage{sectsty}
\allsectionsfont{\sffamily\mdseries\upshape} % (See the fntguide.pdf for font help)
% (This matches ConTeXt defaults)

%%% ToC (table of contents) APPEARANCE
\usepackage[nottoc,notlof,notlot]{tocbibind} % Put the bibliography in the ToC
\usepackage[titles,subfigure]{tocloft} % Alter the style of the Table of Contents
\renewcommand{\cftsecfont}{\rmfamily\mdseries\upshape}
\renewcommand{\cftsecpagefont}{\rmfamily\mdseries\upshape} % No bold!
\DeclareUnicodeCharacter{00A0}{ }

\usepackage{natbib}
%%% END Article customizations

%%% The "real" document content comes below...

\title{Part-based creature recombination}
\author{Sebastian Geschke}
%\date{} % Activate to display a given date or no date (if empty),
         % otherwise the current date is printed 

\begin{document}
\maketitle
\section{Abstract}
Modern games may consider the possibility of allowing the user to modify the morphology of their character. This is no problem as long as the user is only allowed to change scaling, but if the user is allowed to exchange body parts there may occur additional problems in the development process. One problem will be the differing development process, because animating of each creature combination is impossible. Let‘s say we can switch head, torso, arms, two-/four-legged and tail and there are five different heads, torsi, etc. the animator need to make five to the power of six animations. 

An upcoming requirement will be, that the animations of each part of the creature should look the same whichever other creature parts are chosen. 
This paper presents algorithms that can be used to achiev that goal,


\section{Introduction}
This chapter covers basic terminology to character animation. It also includes a basic overview to the technologies used to achiev motion retargetting and part-based animation.

\subsection{Character animation}
Skeletal animation
Character is divided in surface representation (mesh / skin) and a hierarchical set of bones (rig  / skeleton)
Forward kinematics -> to compute the position of the end-effector from specified values for the joint parameters
Inverse kinematics -> determine the joint parameters that provide a desired position of the end-effector

Morph target animation
Animation is stored as a series of vertices
In each key frame the vertices are interpolated between the stored points

Mocap to the left and keyframe to the right
Mocap uses marker / 3d tracking cam to determin the position skeleton joints
Keyframe a curve determines the position of the joints

Adv. Of mocap -> less time to spend in animating only amendments Disadv.: pretty expensive for good soft and hardware
Adv of keframe anim. -> more robust and controlled Disadv: time consuming


\subsection{Part-based recombination}
Creatures are modeled, animated in the classical manner (offline)
Algorithm subdivides the creature into arms, head, legs, tail and torso (offline)
The user can choose whichever creature part he or she wants
Algorithm merges the creature part into one object

\section{Related Work}
Another way to deal with user-created morphologies
Semantics are added to the character(offline)
Semantics for selecting objects (see "fist" "target") and for movement of objects (see "move")

Define context queries to represent an animation (offline)
e.g "move fist to target"
Evaluate context queries to animate the character


- standart werke zu character animation


\section{Algorithm}
Here we show the algorithm presented by \citep{jain2012exploring} with regard to usage in SomeRacer

\section{Conclusion}
In this section a conclusion  will be presented. We will discuss if part-based recombination is a suitable approach for SomeRacer.
\nocite{*}

\bibliographystyle{chicago}
\bibliography{bibliography}

\end{document}
