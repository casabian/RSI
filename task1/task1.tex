% !TEX TS-program = pdflatex
% !TEX encoding = UTF-8 Unicode

% This is a simple template for a LaTeX document using the "article" class.
% See "book", "report", "letter" for other types of document.

\documentclass[11pt, a4paper]{article} % use larger type; default would be 10pt
\usepackage{a4wide}
\usepackage[utf8]{inputenc} % set input encoding (not needed with XeLaTeX)

%%% Examples of Article customizations
% These packages are optional, depending whether you want the features they provide.
% See the LaTeX Companion or other references for full information.

%%% PAGE DIMENSIONS
\usepackage{geometry} % to change the page dimensions
\geometry{a4paper} % or letterpaper (US) or a5paper or....
% \geometry{margin=2in} % for example, change the margins to 2 inches all round
% \geometry{landscape} % set up the page for landscape
%   read geometry.pdf for detailed page layout information

\usepackage{graphicx} % support the \includegraphics command and options

%\usepackage[parfill]{parskip} % Activate to begin paragraphs with an empty line rather than an indent

%%% PACKAGES
\usepackage{booktabs} % for much better looking tables
\usepackage{array} % for better arrays (eg matrices) in maths
\usepackage{paralist} % very flexible & customisable lists (eg. enumerate/itemize, etc.)
\usepackage{verbatim} % adds environment for commenting out blocks of text & for better verbatim
\usepackage{subfig} % make it possible to include more than one captioned figure/table in a single float
% These packages are all incorporated in the memoir class to one degree or another...

%%% HEADERS & FOOTERS
\usepackage{fancyhdr} % This should be set AFTER setting up the page geometry
\pagestyle{fancy} % options: empty , plain , fancy
\renewcommand{\headrulewidth}{0pt} % customise the layout...
\lhead{}\chead{}\rhead{}
\lfoot{}\cfoot{\thepage}\rfoot{}

%%% SECTION TITLE APPEARANCE
\usepackage{sectsty}
\allsectionsfont{\sffamily\mdseries\upshape} % (See the fntguide.pdf for font help)
% (This matches ConTeXt defaults)

%%% ToC (table of contents) APPEARANCE
\usepackage[nottoc,notlof,notlot]{tocbibind} % Put the bibliography in the ToC
\usepackage[titles,subfigure]{tocloft} % Alter the style of the Table of Contents
\renewcommand{\cftsecfont}{\rmfamily\mdseries\upshape}
\renewcommand{\cftsecpagefont}{\rmfamily\mdseries\upshape} % No bold!

\usepackage{natbib}
%%% END Article customizations

%%% The "real" document content comes below...

\title{Task 1 - Paper Review}
\author{Sebastian Geschke}
%\date{} % Activate to display a given date or no date (if empty),
         % otherwise the current date is printed 

\begin{document}
\maketitle

\noindent \textbf{Paper:} Object Recognition from Local Scale-Invariant Features \newline
\textbf{Author:} David G. Lowe \newline
\textbf{Date of publication:} 1999 
\section{Summary}
Scale-invariant feature transform (SIFT) is an algorithm to detect local features in images regardless of changes in its scale, orientation noise or illumination. It solves the problem of correlation-based approaches where object pose and illumination need to be controlled tightly. Template matching is infeasable if the pose is allowed to vary or objects could be occluded.
Alternative to search the complete image for matches is to extract features from a test image. Many feature types such as line segements or grouping of edges have been explored bu they are not a sufficient reliable basis for object recognition. 

To generate the features a staged filtering approach is used. The stages are:
\begin{itemize}
	\item Scale-invariant feature detection
	\item Feature matching and indexing
	\item Cluster identification
	\item Model verification
	\item Outlier detection
\end{itemize}
The first stage of the staged filtering approach to detect the scale-invariant features uses the maxima or minima of a difference-of-Gaussian (DoG) function in the scale space. These points are then used to generate a feature vector. 
The features used to detect the object are extracted from a training image and usually lie on edges, because they are high in contrast.

The feature matching stage uses a Best-bin-first search, that identifies nearest neighbors with low computation time \citep{Beis:1997}. The nearest neighbors are the keypoints with minimum Euclidean distance form the given feature vector. Lowe speedsup his algorithm by rejects all matches, where the distance ratio is greater than 0.8 which results in eliminating 90\% false matches and 5\% correct matches.

At the end of the paper the author mentions recent reasearch in neuroscience show that neurons in inferior temporal cortex respond to shape features. In addition he meintions that SIFT features appear to be almost the same.

As result of the paper Lowe points out that Object Recognition from Scale-invariant feature transform improve on previous approaches with regard to changes in scale, illumination, and local affine transforms.
As further improvement Lowe mentoins the building of 3D structures of the objects using multiple views.

\section{Positive and negative aspect of the paper}
The structure in general was a positive aspect of the paper. The author split up the problem so that the each section describes a different part of the object detection algorithm.

Nevertheless as a negative aspect of the paper I would mention that Lowe shows its research findings without describing his test setup and implementation of SIFT.

\nocite{*}

\bibliographystyle{chicago}
\bibliography{bibliography}

\end{document}
