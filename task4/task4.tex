% !TEX TS-program = pdflatex
% !TEX encoding = UTF-8 Unicode

% This is a simple template for a LaTeX document using the "article" class.
% See "book", "report", "letter" for other types of document.

\documentclass[11pt, a4paper]{article} % use larger type; default would be 10pt
\usepackage{a4wide}
\usepackage[utf8]{inputenc} % set input encoding (not needed with XeLaTeX)

%%% Examples of Article customizations
% These packages are optional, depending whether you want the features they provide.
% See the LaTeX Companion or other references for full information.

%%% PAGE DIMENSIONS
\usepackage{geometry} % to change the page dimensions
\geometry{a4paper} % or letterpaper (US) or a5paper or....
% \geometry{margin=2in} % for example, change the margins to 2 inches all round
% \geometry{landscape} % set up the page for landscape
%   read geometry.pdf for detailed page layout information

\usepackage{graphicx} % support the \includegraphics command and options

%\usepackage[parfill]{parskip} % Activate to begin paragraphs with an empty line rather than an indent

%%% PACKAGES
\usepackage{booktabs} % for much better looking tables
\usepackage{array} % for better arrays (eg matrices) in maths
\usepackage{paralist} % very flexible & customisable lists (eg. enumerate/itemize, etc.)
\usepackage{verbatim} % adds environment for commenting out blocks of text & for better verbatim
\usepackage{subfig} % make it possible to include more than one captioned figure/table in a single float
% These packages are all incorporated in the memoir class to one degree or another...

%%% HEADERS & FOOTERS
\usepackage{fancyhdr} % This should be set AFTER setting up the page geometry
\pagestyle{fancy} % options: empty , plain , fancy
\renewcommand{\headrulewidth}{0pt} % customise the layout...
\lhead{}\chead{}\rhead{}
\lfoot{}\cfoot{\thepage}\rfoot{}

%%% SECTION TITLE APPEARANCE
\usepackage{sectsty}
\allsectionsfont{\sffamily\mdseries\upshape} % (See the fntguide.pdf for font help)
% (This matches ConTeXt defaults)

%%% ToC (table of contents) APPEARANCE
\usepackage[nottoc,notlof,notlot]{tocbibind} % Put the bibliography in the ToC
\usepackage[titles,subfigure]{tocloft} % Alter the style of the Table of Contents
\renewcommand{\cftsecfont}{\rmfamily\mdseries\upshape}
\renewcommand{\cftsecpagefont}{\rmfamily\mdseries\upshape} % No bold!

\usepackage{natbib}
%%% END Article customizations

%%% The "real" document content comes below...

\title{Task 4 - Short Paper Outline}
\author{Sebastian Geschke}
%\date{} % Activate to display a given date or no date (if empty),
         % otherwise the current date is printed 

\begin{document}
\maketitle
\section{Abstract}
This section will give a brief overview to the short paper "Part-based recombination of Characters in SomeRacer" that is going to be written for the course Research Seminar I.

\section{Introduction}
This chapter covers basic terminology to character animation. It also includes a basic overview to the technologies used to achiev motion retargetting and part-based animation.

\subsection{Character animation}
In this chapter a brief introduction to character animation techniques such as key-frame animation and motion capturing

\subsection{Part-based recombination}
Here we adress the need of part-based recombination to achiev the highest possible amount of character setups for the game SomeRacer.

\section{Related Work}
This chapter covers related work in the field of part-based recombination of characters.

- standart werke zu character animation


\section{Algorithm}
Here we show the algorithm presented by \citep{jain2012exploring} with regard to usage in SomeRacer

\section{Conclusion}
In this section a conclusion  will be presented. We will discuss if part-based recombination is a suitable approach for SomeRacer.
\nocite{*}

\bibliographystyle{chicago}
\bibliography{bibliography}

\end{document}
